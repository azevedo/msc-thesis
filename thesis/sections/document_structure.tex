This dissertation's document is organized as follows:

\subsubsection*{State of the Art}

This chapter presents information about known musical notations and a discussion about structure
types for representing music and their pros and cons according to their intended purposes. Also it
presents some of the most relevant projects and tools being developed or used. Finally it presents
information about musical corpora, how it should be built and what it can be used for.

\subsubsection*{ABC::DT and ABC processing tools}

This chapter presents the three stages comprising an \abcpt{}'s internal structure. As well as the
implementation of \abcdt{}, a Perl embedded \ac{DSL} which aims to facilitate the creation of new
\abcpt{}s.

\subsubsection*{ABC::DT by example}

This chapter presents examples of tools created using \abcdt{}, thus demonstrating how easily a
(simple and compact) tool or some occasional processing can be made.

\subsubsection*{Test and Evaluation}

A test and evaluation are made to one \abcpt{} developed within this dissertation writing period.
The goal is to help analyze the \abcpt{}'s behavior and to support some claims that are made
throughout this dissertation.

\subsubsection*{Conclusions and Future Work}

This chapter presents a recapitulation and an assessment of what was discussed throughout the
dissertation. Some possible future work is described.

\subsubsection*{Appendix}

This chapter presents tables and images referenced throughout the dissertation.

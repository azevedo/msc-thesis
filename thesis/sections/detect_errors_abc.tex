\abc{} is a textual music notation, therefore it is very common for an \abc{} score to have
syntactical errors, such as, having more beats in a measure than it can hold.

There are three kinds of behavior when facing an error: correct it immediately (e.g.: insert a bar
when it's missing); warn the user of the error's existence; and comment the error and annotate a
\emph{FIXME} comment so that the user can locate and fix the error manually.

Due to time limitations, only one behavior is adopted in \detecterrorsabc{}, which is to warn the
user. So, for each file, it detects errors and produces an output with error messages along with the
voice and measure number where they occurred.

\detecterrorsabc{} will expose the following errors:

\begin{description}
  \item[Incomplete/Overflowing measure] \hfill \\
    A measure is a segment of time defined by a given number of beats which is delimited by a
    \emph{bar} element. The number of beats in a measure is determined by the \emph{Meter (M:)}
    previously defined.

    The first metrically complete measure within a score is the first measure. When the score begins
    with an anacrusis (an incomplete measure at the head of a score), the first measure is the
    following measure.

    So, the number of beats (the length of all notes and rests) in a measure (except if it is an
    anacrusis) must be equal to that measure's defined length.

  \item[Last \abc{} element per voice isn't a bar] \hfill \\
    \abc{} allows a score to not having a \emph{bar} at the end of a voice, however, it isn't
    considered a good practice in modern music notation.

    Therefore, a voice must finish with a \emph{bar} element, in other words, the last
    \abcelement{}, except \emph{eoln}, for a voice has to be a \emph{bar}.

  \item[Different number of measures per voice] \hfill \\
    All voices must have the same number of measures.

  \item[Different key definitions per measure] \hfill \\
    In modern music, there are certain properties that apply to many elements simultaneously, for
    instance, in a multi-voice score, the second measure is the second measure for all voices, as
    well as the key, among others. However, in \abc{}, that assumption may be ignored.

    Thus, for each measure, the key must be the same for all voices.

\end{description}

\subsection*{Algorithm}

\detecterrorsabc{}'s algorithm consists of 2 stages that are applied to each tune.

An algorithmic description is made in algorithm \ref{alg:detecterrorsabc}.\\

For each tune:

\begin{enumerate}
  \item \textbf{Retrieve data and detect incomplete/overflowing measures}

  In this stage, the tune is processed by \dt{}, in which each voice's \emph{last \abc {} element},
  \emph{number of measures} and the \emph{key per measure} are stored to be used in the following
  stage.

  It also stores the \emph{current measure's real length} in order to detect incomplete/overflowing
  measures. In order to accomplish the latter, when the \emph{bar} element is found, it compares the
  \emph{current measure's real length} with the current measure's defined length.

  The set of \abcdtrules{} is shown in table \ref{tab:detect_errors_abc_fst_stage}.

  \begin{center}
    \begin{table}[h]
      \begin{tabular}{|p{2cm}|p{6cm}|p{7.5cm}|}
        \hline
        Actuator & Transformation (Perl) & Notes\\
        \hline
        \hline
        \emph{in\_tune} & update\_data(\{\}); & Local function that sets the current element as the
        current voice's last \abc{} element.
        \\
        \hline

        \hline
        \emph{note} & update\_data(\{meas\_dur => 1, n\_meas => 1\}); & Local function that sets the
        current element as the current voice's last \abc{} element. When \emph{meas\_dur} is 1, it
        increments the current measure's real length with the element's value. When \emph{n\_meas}
        is 1, it updates the current voice's number of measures.
        \\
        \hline

        \hline
        \emph{rest} & update\_data(\{meas\_dur => 1, n\_meas => 1\}); &
        \\
        \hline

        \hline
        \emph{mrest} & update\_data(\{key => 1\}); & Local function that sets the current element as
        the current voice's last \abc{} element. When \emph{key} is 1, it updates the key for all
        measures that \emph{mrest} covers.
        \\
        \hline

        \hline
        \emph{bar} & \$ret .= check\_measure\_length(); update\_data(\{n\_meas => 1, key => 1\}); &
        \emph{check\_measure\_length} is a local function that detects if the current measure is
        incomplete or overflowing and returns an error message in case it finds one.
        \\
        \hline

        \hline
        \emph{eoln} & q\{\}; & The end of a line won't be set as a voice's last element.
        \\
        \hline
      \end{tabular}
      \caption{\abcdtrules{} for \detecterrorsabc{}'s first stage}
      \label{tab:detect_errors_abc_fst_stage}
    \end{table}
  \end{center}

  \item \textbf{Detect the remaining syntactical errors}

  It detects if the tune's last \abc{} element for all voices is a \emph{bar} and if the number of
  measures is the same for all voices. Error messages are produced in case errors are found.

  If no errors were found until this moment, then the detection for different key definitions per
  measure proceeds.
\end{enumerate}

\begin{algorithm}[H]
  \KwIn{abc\_tunes}
  \ForAll{$ tune \in abc\_tunes $}{
    $dt(tune,$ rules from table \ref{tab:detect_errors_abc_fst_stage}$)$ \hfill //1)\\
    $res \gets res$ ++ $detect\_remaining\_errors()$ \hfill //2)\\
  }
  \Return{$res$}
  \caption{\detecterrorsabc{}'s algorithm}
  \label{alg:detecterrorsabc}
\end{algorithm}

\subsection*{Usage}

Listing \ref{lst:detecterrorsabcman} shows \detecterrorsabc{}'s manual.\\

\lstinputlisting[caption={\detecterrorsabc{}'s manual},label={lst:detecterrorsabcman},captionpos=t,abovecaptionskip=-\medskipamount]{misc/detect_errors_manual.tex}

Listing \ref{lst:detecterrorsabcbyexample} shows a usage example for \detecterrorsabc{}. It reads
the tune \textbf{100\_errors.abc} (listing \ref{lst:100errors}) and the output is shown in listing
\ref{lst:detect_errors_output}.\\

\begin{lstlisting}[caption={\detecterrorsabc{} by example},label={lst:detecterrorsabcbyexample},captionpos=t,abovecaptionskip=-\medskipamount]
detect_errorsabc 100_errors.abc
\end{lstlisting}

\lstinputlisting[caption={100.abc with errors},label={lst:100errors},captionpos=t,abovecaptionskip=-\medskipamount]{misc/100_errors.tex}

\lstinputlisting[caption={\detecterrorsabc{}'s output},label={lst:detect_errors_output},captionpos=t,abovecaptionskip=-\medskipamount]{misc/detect_errors_output.tex}

\subsection*{Textual Musical Notation}

All music needs to be written before read, comprehended or performed by any musician. To make it
possible, a notation system has been developed that provides musicians with the information
necessary to reproduce it as the composer wanted.

The notation consists in any system that represents audible music through written symbols. The use
of symbolic and abstract formats improves music reasoning, as it gives the composer a greater
freedom to express his music and provides easier readability to the performer.

As computers were introduced to the world of music, a variety of file formats and textual notations
emerged in order to describe music, such as, \abc{}~\cite{abcnotation:Online},
LilyPond~\cite{lilypond:Online} or MusicXML~\cite{musicxml:Online}.

\abc{} is used as the base notation throughout this dissertation.

\subsection*{\unix{} Metaphor}

In the 1970s the \ac{OS} \unix{} was born and with it a new philosophy\cite{raymond2004art} based on
the principle of creating simple, yet capable and efficient programs, which tackle only one problem
at a time.

The system's interface is the command line, thus making the work method very powerful and flexible
as it enables the automatic execution of commands. Moreover, commands handle text streams as a
universal type, allowing programs to be chained.

% There are some \unix{} commands whose behavior can be adapted:
% \begin{description}
%   \item[grep:]
%     Prints the lines of a file matching a pattern;\\
%     It could print melodic sequences matching a pattern. The pattern could be a sequence of melodic
%     intervals or rhythms;
%   \item[diff:]
%     Compares files line by line;\\
%     It could compare files voice by voice;
%   \item[\ldots]
% \end{description}


In order to facilitate the development of new \unix{} commands, \unix{} creators built a new
language (C).

\begin{quotation}
  \small\textit{\unix{} is simple. It just takes a genius to understand its simplicity.}
  \begin{flushright}
    Dennis Ritchie
  \end{flushright}
\end{quotation}

When moving to the music world, the goal is to build simple music commands, using a universal music
stream type - \abc{} -, creating a development language for conceiving music commands and exercising
command compositionality.

%% CLAIM %%
Each command - an \abcpt{} - assists in the solving of music related problems. For instance,
questions like \textit{"How many times does that happen in Beethoven's sonatas?"}, or \textit{"I
wish I could extract these parts from this score and transpose them up a major second"} could be
easily answered.

%% CLAIM %%
As \unix{}, with its universal interface - the text stream -, an \abcpt{} uses \abc{} as its
universal interface. Being text as well, an \abcpt{} can be chained with others.

%% CLAIM %%
There are many \unix{} commands whose functionality can be mapped to an \abcpt{}. This is true mainly
because of the textual nature of their input. Here are some \unix{} commands that could be mapped to
\abcpt{}s: musical cat, paste, wc, grep, diff, cut, join, sed, ...

\subsection*{\abc{} toolkit}

%% CLAIM %%
Presently, there is a lack of music notation general processing tools, particularly for \abc{}. So
the main goal is to build an \abc{} \ac{OS}, i.e. a system that provides a set of \abcpt{}s that
deal with real \abc{} and aid in musical tasks like analysis, composition, studying, ...

\subsection*{\abcdt{}}

%% CLAIM %%
In order to easily build simple and compositional (in a \unix{} filters' meaning) \abcpt{}s, a
system capable of precisely specifying how an \abc{} score should be processed is needed. The same
necessity appeared to \unix{}'s developers while developing it, from which the C language was built.
Therefore, a rule-based \ac{DSL}~\cite{kosar2010comparing,kosar2008preliminary} was created - \abcdt{}.

\subsection*{Application Areas}

In order to better understand the benefits of the toolkit being proposed, some real life activities
where those tools can help make a difference are described next:

\begin{description}
  \item[Musical Wikis] \hfill \\
    Wikis that deal with the edition of musical scores. E.g.: Wiki::Score.
  \item[Cultural and cooperative volunteering] \hfill \\
    In environments like Wikis where the edition of documents happens concurrently in cooperation
    with many elements. E.g.: Wiki::Score.
  \item[Score transcription] \hfill \\
    Often errors occur while manually transcribing a score and those errors are not easily detected.
  \item[Music learning] \hfill \\
    Custom tools may be created with the purpose of supporting tasks such as studying and
    rehearsing.
  \item[Musical analysis and composition] \hfill \\
    E.g.: Through the detection of certain patterns in specific musical style it is possible to
    assess if a composition uses any feature of that style;\\
    E.g.: Automatic classification of scores;\\
    E.g.: Verification of a score's authorship. In the same way that it is possible to assess the
    probability of an author having written a certain text through the type of vocabulary used.\\
    E.g.: Generation of scores that follow strict structural rules, such as the canon or the fugue.
\end{description}

The next section presents the work developed in the context of this dissertation, enumerating the
design goals that guided the project's development. Section \ref{sec:case_studies} introduces the
case studies which served as motivation to this work and Section \ref{sec:document_structure}
presents the structure of the document, including a summary of each chapter.

In order to calculate the difference between what is considered a pattern, assess what is expected,
calculate similarities between scores or generate statistics there must exist some example cases.

Those example cases are called corpus and in this dissertation's case it is a specific corpus (a
musical corpus) which contains rich metadata regarding musical scores. The knowledge generated by
the analysis of the corpus may be shared by many tools through a richer combination of tools.

The corpus can be used as testing material for the toolkit, for instance, a tool that validates an
\abc{} score's syntax needs either flawless examples or examples with deliberately typed errors to
guarantee that it works as it's supposed to. Also, it can be used to train systems that learn from
data, for instance, a system that is trained with a set of scores in order to learn how to identify
certain music aspects, such as the style.

% Other uses for the corpus include:
% \begin{description}
%   \item[Testing material for the toolkit] \hfill \\
%     A tool that validates syntax needs either flawless examples or examples with deliberately typed errors
% to guarantee that it works as it's supposed to. 
%   \item[Training systems that learn from data] \hfill \\
%     A system that is trained with a set of scores in order to learn how to identify certain music
%     aspects, such as the style.
% \end{description}

\subsection{Building corpora}

This phase, according to existing literature on building corpora~\cite{Atkins1992,
wynne2005developing,teseandre} (plural of corpus), consists of planning the whole process and
annotating them.

\subsubsection*{Planning}

In this step decisions have to be made so that the remaining steps may take place. They consist in
defining the quantity of scores that will be added to the corpus, selecting the scores that should
be added (according to their use and availability), defining the intermediary formats and
conventions to be used in the processing pipeline, defining if and what annotations should be
included in the corpus and finally, defining in which formats the corpus should be available and how
should the analysis tools interface with them.

% \begin{itemize}
%   \item Define the quantity of scores that will be added to the corpus;
%   \item Select the scores that should be added, according to their use and availability;
%   \item Define the intermediary formats and conventions to be used in the processing pipeline;
%   \item Define if and what annotations should be included in the corpus;
%   \item Define in which formats the corpus should be available and how should the analysis tools
%     interface with them.
% \end{itemize}

% \subsubsection*{Getting permissions}
% 
% Scores will be gathered from public domain as they are not subject to copyright.

\subsubsection*{Gathering scores}

\abc{} notation has become very popular since its introduction, and nowadays thousands of tunes
exist in electronic format. Scores, as in a music consisting of multiple voices, exist in a lesser
number because the features that allow the writing of polyphony were only added to the standard much
later, yet it is an ever growing culture. However, a previous parsing and reformatting might still
be needed in order to process them efficiently.

\subsubsection*{Annotating scores}

In order to improve the usefulness of a corpus for a richer and more rigorous statistical analysis,
it might be subject to the process of annotation. It consists in applying some sort of structural
representation to act as a blue print of the original text and to provide additional interpretative
information.

% Yet, the use of annotations is not always the best approach. Susan Hunston, in \textit{Corpora in
% Applied Linguistics}~\cite{hunston2002corpora} states that the kind of questions that are usually
% asked to a corpus tend to be limited rather than the questions that can be asked. This happens
% because the categories used to annotate a corpus are frequently determined before any analysis is
% done. Thus, as stated before, a careful planning has to be made.


\subsection{What can be analysed} 

It's desired that a set of tools for statistical calculation is built. To make that possible a large
set of corpora must be built, so that statistical analysis and hypothesis
testing\footnote{Hypothesis testing is the use of statistics to determine the probability that a
given hypothesis is true.} can be performed and from the results extract valuable information.

The corpus may be used for finding sets of vertical patterns that occur in a large number of scores
in the corpus\cite{Conklin2002}, measuring rhythmic similarity (the repetitive nature of the music)
with manual annotations to the corpus\cite{antonopoulos2007music}, identifying trends and changes
throughout a historical time period through cluster analysis\cite{albrechtemergence}, among many
other uses.

% Some examples of what can be analysed in musical corpora:
% 
% \begin{itemize}
%   \item Find sets of vertical patterns that occur in a large number of scores in the
%     corpus\cite{Conklin2002};
%   \item Find significant statistical differences between melodies of folk music from different
%     countries\cite{chai2001folk};
%   \item Display multiple excerpts from a collection of scores, such as all of the cadences, without
%     human intervention and editing\cite{Knopke};
%   \item Measure rhythmic similarity - the repetitive nature of the music - with manual annotations
%     to the corpus\cite{antonopoulos2007music};
%   \item Study norms of behavior for documenting relationships between accents and harmonic structure
%     in common-practice music, and the role of notational variants in identifying scribes and
%     composers\cite{Ariza};
%   \item Identify trends and changes throughout a historical time period through cluster
%     analysis\cite{albrechtemergence}.
% \end{itemize}

\subsection{Existing Corpora} 

There are many existing musical corpus available in the Internet. A large corpus will be assembled
ranging from \abc{} corpus, to \midi{}, MusicXML and still LilyPond. 

The corpus format may vary depending on what the tools can read and process, for instance, if
\midi{} transformations are implemented then a \midi{} corpus has to exist as well. As this
dissertation's focus is \abc{}, the main format will also be \abc{}.

Here are some of the websites and packages from where they'll be gathered:

\begin{description}
  \item[http://abcnotation.com/browseTunes] \hfill \\
    Around 350,000 tunes available as \abc{} or \midi{} sound files;
  \item[http://thesession.org/] \hfill \\
    Around 11,000 tunes available as \abc{} or \midi{} sound files;
  \item[http://moinejf.free.fr/abc/index.html] \hfill \\
    \abc{} organ pieces;
  \item[http://www.classicalarchives.com] \hfill \\
    Around 14,000 \midi{} sound files;
  \item[http://abc.sourceforge.net/NMD/] \hfill \\
    Around 1000 \abc{} files;
  \item[Music21 corpus package] \hfill \\
    A collection of approximately 10,000 works including a complete collection of the Bach Chorales,
    numerous Beethoven String Quartets, and examples of Renaissance polyphony. The corpus includes
    \abc{}, MusicXML and Kern files.
  \item[http://wiki-score.org/] \hfill \\
    Modern editions written in \abc{} of unknown works buried in music archives.
  \item[http://www.mutopiaproject.org/] \hfill \\
    Sheet music editions of classical music in LilyPond.
\end{description}

\subsection{Summary}

A musical corpus will be built according to the needs of the toolkit. The initial need is for a
toolkit that reads and processes \abc{} notation, so the main focus will be to build an \abc{}
corpus.

A careful planning on how to build the corpus plays an important part on determining the quality and
quantity of tasks that can make use of it. Such planning strongly affects the final results an
\abcpt{} can produce.

This tool, is similar to \unix{}'s wc, in the sense that it prints voices, measures and
notes/pitches per voice counts for each \abc{} file.

This tool generates a textual summary of the counts made.
For each tune it prints:
\begin{itemize}
  \item number of voices
  \item for each voice:
  \begin{itemize}
    \item total number of notes
    \item total number of measures
    \item number of occurrences of a certain note (pitch)
  \end{itemize}
\end{itemize}

\subsection*{Algorithm}

\wcabc{}'s algorithm consists in processing each tune with \dt{} in order to produce the desired
counts. In the end an output is generated with the produced data.

An algorithmic description is made in algorithm \ref{alg:wcabc}.

The voice count is updated when the \emph{voice} element is found, the note and pitch counts are
updated when the \emph{note} element is found and the measure count is updated when the \emph{bar}
element is found. The set of \abcdtrules{} is shown in table \ref{tab:wc_abc_rules}.

\begin{center}
  \begin{table}[h]
    \begin{tabular}{|p{2cm}|p{5cm}|p{7.5cm}|}
      \hline
      Actuator & Transformation (Perl) & Notes\\
      \hline
      \hline
      \emph{V:} & update\_voice\_count(); & Local function that increments the voice count when a
      new voice is found.
      \\ \cline{1-2}
      \hline

      \hline
      \emph{note} & update\_note\_count(); & Local function that increments the note and pitch
      count. For the pitch name, it uses \abcdt{}'s function \emph{get\_pitch\_name()}.
      \\ \cline{1-2}
      \hline

      \hline
      \emph{bar} & update\_measure\_count(); & Local function that increments the measure count
      according to the bar number.
      \\ \cline{1-2}
      \hline
    \end{tabular}
    \caption{\abcdtrules{} for \wcabc{}}
    \label{tab:wc_abc_rules}
  \end{table}
\end{center}

\begin{algorithm}[h]
  \KwIn{abc\_tunes}
  \ForAll{$ tune \in abc\_tunes $}{
    $dt(tune,$ rules from table \ref{tab:wc_abc_rules}$)$\\
  }
  $res \gets create\_output()$\\
  \Return{$res$}
  \caption{\wcabc{}'s algorithm}
  \label{alg:wcabc}
\end{algorithm}

\subsection*{Usage}

Listing \ref{lst:wcabcman} shows \wcabc{}'s manual.\\

\lstinputlisting[caption={\wcabc{}'s manual},label={lst:wcabcman},captionpos=t,abovecaptionskip=-\medskipamount]{misc/wc_manual.tex}

Listing \ref{lst:wcabcbyexample} shows a usage example for \wcabc{}. It reads the tune generated by
\pasteabc{} (listing \ref{lst:verbum_s1_p1_p3_pasted}) and the output is shown in listing
\ref{lst:wc_output}.\\

\begin{lstlisting}[caption={\wcabc{} by example},label={lst:wcabcbyexample},captionpos=t,abovecaptionskip=-\medskipamount]
wc_abc 101_103.abc
\end{lstlisting}

\lstinputlisting[caption={\wcabc{}'s output},label={lst:wc_output},captionpos=t,abovecaptionskip=-\medskipamount]{misc/verbum_s1_p1_p3_pasted_wc.tex}

\wcabc{} reports that there are 2 voices. Voice with \emph{id} 1 has 8 measures, a total of 18 notes
and 6 \emph{G}'s, 4 \emph{F}'s, 3 \emph{A}'s, 3 \emph{E}'s, 2 \emph{B}'s and 1 \emph{D}.  The
interpretation for voice with \emph{id} 3 is analogous.

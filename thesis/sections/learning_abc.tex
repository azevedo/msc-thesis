When there is a multi-voice score, like a four part choir, it is important to, for instance, the
Soprano to be able to study her part individually. Sometimes there's the need to hear all the other
parts except hers, so that she may know what the rest is going to sound. Other times the opposite is
what is needed.

The \learningabc{} tool generates two \abc{} scores whose goal is to help musicians in individual
rehearsal of multi-voice music for studying purposes. One reduces the volume of a particular voice
and the other increases the volume of a particular voice and reduces the volume of the remaining
voices.

\subsection*{Algorithm}

\learningabc{}'s algorithm consists of 3 stages that are applied to each tune.

An algorithmic description is made in algorithm \ref{alg:learningabc}.\\

For each tune:
\begin{enumerate}
  \item \textbf{Retrieve voice data}

  In this stage, the tune is processed by \dt{}, in which each voice's name, channel and program
  (instrument) are stored to be used in the following stages.

  The set of \abcdtrules{} is shown in table \ref{tab:learning_abc_fst_stage}.

  \begin{center}
    \begin{table}[h]
      \begin{tabular}{|p{3cm}|p{5cm}|p{7.5cm}|}
        \hline
        Actuator & Transformation (Perl) & Notes\\
        \hline
        \hline
        \emph{V:} & store\_channel\_and\_name(); & Local function that stores each voice's channel
        and name.
        \\
        \hline

        \hline
        \emph{MIDI::program} & store\_program(); & Local function that stores each voice's program.
        \\
        \hline
      \end{tabular}
      \caption{\abcdtrules{} for \learningabc{}'s first stage}
      \label{tab:learning_abc_fst_stage}
    \end{table}
  \end{center}

  \item \textbf{All but one}

  In \abc{}, it is possible to add commands to control audio properties through the use of \midi{}
  directives (\emph{\%\%MIDI}, followed by different parameters) that \abctomidi{} recognizes.

  In this stage, the tune is processed by \dt{}, in which a \midi{} directive to reduce the volume
  of the voice is inserted. To be more precise, a change-volume \midi{} directive (\emph{\%\%MIDI
  control 7 NewVolume}, where \emph{NewVolume} is a number between (0-127)- is appended after the
  voice definition.

  In the end, the processed tune is written to a new \abc{} file.

  The set of \abcdtrules{} is shown in table \ref{tab:learning_abc_snd_stage}.

  \begin{center}
    \begin{table}[h]
      \begin{tabular}{|p{2.25cm}|p{6.75cm}|p{6.5cm}|}
        \hline
        Actuator & Transformation (Perl) & Notes\\
        \hline
        \hline
        \emph{V:\$req\_voice} & toabc() . "\%\%MIDI control 7 \$min\_volume\textbackslash{}n"); &
        \emph{\$req\_voice} keeps the voice requested when calling \learningabc{}.
        \\
        \hline
      \end{tabular}
      \caption{\abcdtrules{} for \learningabc{}'s second stage}
      \label{tab:learning_abc_snd_stage}
    \end{table}
  \end{center}

  \item \textbf{Just one}

  In this stage, the tune is processed by \dt{}, in which, for each voice, three \midi{} directives
  are inserted after the \emph{X:} statement. The first is a select-channel directive
  (\emph{\%\%MIDI channel Channel}, where \emph{Channel} is a number between (1-16)), the second a
  select-program directive (\emph{\%\%MIDI program Channel Program}, where \emph{Program} is the
  instrument (0-127) for channel \emph{Channel}) and the third is a change-volume directive to
  reduce or increase the voice's volume. Furthermore, the select-channel directive is also appended
  to the voice statement so that \abctomidi{} can make the association between the voice and the
  channel when reproducing.

  In the end, the processed tune is written to a new \abc{} file.

  The set of \abcdtrules{} is shown in table \ref{tab:learning_abc_trd_stage}.

  \begin{center}
    \begin{table}[h]
      \begin{tabular}{|p{1.5cm}|p{7.5cm}|p{6.5cm}|}
        \hline
        Actuator & Transformation (Perl) & Notes\\
        \hline
        \hline
        \emph{X:} & set\_volume(); & Local function that sets the volume by appending the three
        \midi{} directives mentioned before for each voice.
        \\
        \hline

        \hline
        \emph{V:} & toabc() . "\%\%MIDI channel
        \$voice\_channel\{\$c\_voice\}\{channel\}\textbackslash{}n"; & Appends the select-channel
        directive after its corresponding voice.
        \\
        \hline
      \end{tabular}
      \caption{\abcdtrules{} for \learningabc{}'s third stage}
      \label{tab:learning_abc_trd_stage}
    \end{table}
  \end{center}
\end{enumerate}

\begin{algorithm}[h]
  \KwIn{abc\_tunes}
  \ForAll{$ tune \in abc\_tunes $}{
    $dt(tune,$ rules from table \ref{tab:learning_abc_fst_stage}$)$ \hfill //1)\\
    $just\_one \gets dt(tune,$ rules from table \ref{tab:learning_abc_snd_stage}$)$ \hfill //2)\\
    $write\_to\_file(just\_one)$ \hfill //2)\\
    $all\_but\_one \gets dt(tune,$ rules from table \ref{tab:learning_abc_trd_stage}$)$ \hfill //3)\\
    $write\_to\_file(all\_but\_one)$ \hfill //3)\\
  }
  \Return{}
  \caption{\learningabc{}'s algorithm}
  \label{alg:learningabc}
\end{algorithm}

\subsection*{Usage}

Listing \ref{lst:learningabcman} shows \learningabc{}'s manual.\\

\lstinputlisting[caption={\learningabc{}'s manual},label={lst:learningabcman},captionpos=t,abovecaptionskip=-\medskipamount]{misc/learning_manual.tex}

Listing \ref{lst:learningabcbyexample} shows a usage example for \learningabc{}. It reads tune
\textbf{100.abc} (listing \ref{lst:verbum_s1}) and the output is shown in listing
\ref{lst:100_all_but_tenor} and \ref{lst:100_just_tenor}.\\

\begin{lstlisting}[caption={\learningabc{} by example},label={lst:learningabcbyexample},captionpos=t,abovecaptionskip=-\medskipamount]
learning_abc -v=Tenor -min=25 100.abc
\end{lstlisting}

\begin{center}
  \begin{minipage}{.49\textwidth}
    \lstinputlisting[caption={100.abc},label={lst:verbum_s1},captionpos=t,abovecaptionskip=-\medskipamount]{misc/verbum_s1.tex}
  \end{minipage}
  \hfill
  \begin{minipage}{.49\textwidth}
    \lstinputlisting[caption={100\_all\_but\_Tenor.abc},label={lst:100_all_but_tenor},captionpos=t,abovecaptionskip=-\medskipamount]{misc/100_all_but_Tenor.tex}
  \end{minipage}
\end{center}

Note the \midi{} command \texttt{\%\%MIDI control 7 25} after \texttt{V:3} in listing
\ref{lst:100_all_but_tenor}. That way, voice "Tenor" is going to be attenuated when \abctomidi{}
reproduces the score.\\

\lstinputlisting[caption={100\_just\_Tenor.abc},label={lst:100_just_tenor},captionpos=t,abovecaptionskip=-\medskipamount]{misc/100_just_Tenor.tex}

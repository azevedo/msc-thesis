Case studies help understand the origin of the problem and the problem itself, serving as a guide to
the development of a solution.

\subsection*{Wiki::Score}

Wiki::Score\cite{almeida2012wiki}\footnote{http://wiki-score.org/} is a platform similar to the
Wikipedia for cooperative editing of large scale music scores (eg. operas, symphonies, cantatas,
etc). Wiki::Score is a Wiki which, using the ABC notation for music representation, is primarily
intended for publishing modern editions of unknown works buried in music archives. It has emerged
from experience in the lab sessions of the Computing for Musicology course of the Music degree of
the University of Minho.

Being a Wiki anyone can edit a part and submit it, moreover it allows concurrent editions on the
same source. This makes Wiki::Score prone to having many errors in its scores.

Wiki::Score was the original motivation for this toolkit, therefore many of the requirements defined
came from the shortcomings it presented. It also serves as a mean to test and validate the tools
developed.

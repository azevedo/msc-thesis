\documentclass[main.tex]{subfiles}
\begin{document}

This thesis will present work developed mainly in the context of Wiki::Score, a Wiki intended for
publishing modern editions written in the Abc notation of unknown works buried in music archives. 

Given the context presented, a toolkit will be built. A set of tools in which each tool individually
solves a specific problem and whose results may be articulated with others.\\
It's main goal is the increase on the abstraction and granularity of musical information applied to
musicology problems. Hence from simple musical information, knowledge may be created, for instance,
represented by effective views of that information.

\subsection{Design Goals}

It's important that a set of design goals is established to efficiently guide the toolkit's
development:

\subsubsection{Toolkit}

The tools will cover all the problems associated with the processing of textual musical notation:

\begin{description}
  \item \textbf{Automatic validation of the notation and error detection} \hfill\\
    When writing Abc, since it is ASCII based, it is relatively easy to write errors on the score.
    One of the features that is required for this toolkit is for it to be able to detect those
    errors thus validating the notation in terms of syntax, lexicon, among others that are listed
    below:
    \begin{itemize}
      \item \textbf{syntax}\\
        e.g.: beats per measure, different key definitions in voices;
      \item \textbf{lexicon}\\
        e.g.: inappropriate use of symbols;
      \item \textbf{misalignment}\\
        e.g.: voice disparity due to transcription errors;
      \item \textbf{suspicious use of musical symbols}\\
        e.g.: use of chords out of context in a certain musical style;
        e.g.: use of notes out of vocal range;
    \end{itemize}
  \item \textbf{Error fixing} \hfill\\
    Since it can detect erros it's only natural that they can be fixed, either by automatically
    setting the appropriate fix or by suggesting one to the user. Typically the errors fixed will be
    of syntactic or lexical nature.
    \begin{itemize}
      \item \textbf{syntactic}\\
        e.g.: when there are many consecutive rests, then a suggestion is made to join them by
        measure;
        e.g.: when the key is changed in only one voice, then that change is reproduced in all
        voices;
      \item \textbf{lexical}\\
        e.g.: when there's an orthographic or a capitalization error in reserved words, then it is
        automatically fixed;
    \end{itemize}
\end{description}

Furthermore, it must:

\begin{description}
  \item \textbf{Be open-source} \hfill\\
    The source code will be available to the general public therefore it will follow community
    practices and will be installable and usable as a third-party tool.
    The existence of an open-source community allows the exchange of information and ideas between
    developers and users. Interesting things can come out of a discussion in this mean, like new
    features or a solution to a certain problem.
  \item \textbf{Follow the Unix philosophy} \hfill\\
    Tools that tackle a single problem and can be integrated with others to solve bigger problems.\\
    There are many Unix programs whose functionality can be mapped to this toolkit. This is true
    mainly because of the textual nature of their input. Here are some Unix programs that could be
    mapped to musical filters: musical grep, diff, cut, paste, cat, join, sed.
\end{description}

\subsubsection{Musical Corpora}

In order to do statistical analysis there must be:
\begin{description}
  \item \textbf{Musical corpus comprised of musical scores} \hfill\\
    This corpus will be used as a source of data for the analysis.
  \item \textbf{Build tools for statistical calculation} \hfill\\
    Each tool will output some statistical information.
\end{description}

\subsubsection{Musical Information Representation}

The internal representation of a score must:
\begin{description}
  \item \textbf{Keep the original order of the score symbols} \hfill\\
    This is an obvious goal since almost every task needs to know the exact order of a score to
    produce anything useful.
  \item \textbf{Hold sufficient musical information to rebuild the score as it was} \hfill\\
    This way a score can easily be outputted as it was originally.
  \item \textbf{Facilitate the application of scripting} \hfill\\
    This means the internal representation can be represented into a structure that is easily
    accessible to a language like Perl.
\end{description}

\subsubsection{Musical Information Visualization}

There are always different forms of displaying musical information to a user. Be it a graph, a
drawing, some sort of symbol, the actual score or even simple text, the output must always transmit
knowledge to the user so that a conclusion can be taken from it.
Thus the output of a tool must:

\begin{description}
  \item \textbf{Have an appropriate format (graphical, textual, other)} \hfill\\
    So that the user can make the most out of the tool results.
  \item \textbf{Be easy to comprehend} \hfill\\
    The results cannot be cryptic, otherwise the user will not understand.  
  \item \textbf{Reveal some feature of the music} \hfill\\
    If an analysis is made to a score some sort of feature - hidden or not - must be revealed.
\end{description}

\subsection{Case Studies}

Case studies help understand the origin of the problem and the problem itself, serving as a guide to
the development of a solution.

\subsubsection{Wiki::Score}

Wiki::Score\cite{almeida2012wiki}\footnote{http://wiki-score.org/} is a platform similar to the
Wikipedia for cooperative editing of large scale music scores (eg. operas, symphonies, cantatas
etc). Wiki::Score is a Wiki which, using the ABC notation for music representation, is primarily
intended for publishing modern editions of unknown works buried in music archives. It has emerged
from experience in the lab sessions of the Computing for Musicology course of the Music degree of
the University of Minho.

Being a Wiki anyone can edit a part and submit it, moreover it allows concurrent editions on the
same source.

Wiki::Score was the original motivation for this toolkit, therefore many of the requirements defined
came from the shortcomings it presents. It also serves as a mean to test and validate the tools
developed.

\end{document}
